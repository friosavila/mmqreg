% Options for packages loaded elsewhere
\PassOptionsToPackage{unicode}{hyperref}
\PassOptionsToPackage{hyphens}{url}
\PassOptionsToPackage{dvipsnames,svgnames,x11names}{xcolor}
%
\documentclass[
  authoryear,
  preprint,
  1p]{elsarticle}

\usepackage{amsmath,amssymb}
\usepackage{iftex}
\ifPDFTeX
  \usepackage[T1]{fontenc}
  \usepackage[utf8]{inputenc}
  \usepackage{textcomp} % provide euro and other symbols
\else % if luatex or xetex
  \usepackage{unicode-math}
  \defaultfontfeatures{Scale=MatchLowercase}
  \defaultfontfeatures[\rmfamily]{Ligatures=TeX,Scale=1}
\fi
\usepackage{lmodern}
\ifPDFTeX\else  
    % xetex/luatex font selection
\fi
% Use upquote if available, for straight quotes in verbatim environments
\IfFileExists{upquote.sty}{\usepackage{upquote}}{}
\IfFileExists{microtype.sty}{% use microtype if available
  \usepackage[]{microtype}
  \UseMicrotypeSet[protrusion]{basicmath} % disable protrusion for tt fonts
}{}
\makeatletter
\@ifundefined{KOMAClassName}{% if non-KOMA class
  \IfFileExists{parskip.sty}{%
    \usepackage{parskip}
  }{% else
    \setlength{\parindent}{0pt}
    \setlength{\parskip}{6pt plus 2pt minus 1pt}}
}{% if KOMA class
  \KOMAoptions{parskip=half}}
\makeatother
\usepackage{xcolor}
\setlength{\emergencystretch}{3em} % prevent overfull lines
\setcounter{secnumdepth}{5}
% Make \paragraph and \subparagraph free-standing
\ifx\paragraph\undefined\else
  \let\oldparagraph\paragraph
  \renewcommand{\paragraph}[1]{\oldparagraph{#1}\mbox{}}
\fi
\ifx\subparagraph\undefined\else
  \let\oldsubparagraph\subparagraph
  \renewcommand{\subparagraph}[1]{\oldsubparagraph{#1}\mbox{}}
\fi


\providecommand{\tightlist}{%
  \setlength{\itemsep}{0pt}\setlength{\parskip}{0pt}}\usepackage{longtable,booktabs,array}
\usepackage{calc} % for calculating minipage widths
% Correct order of tables after \paragraph or \subparagraph
\usepackage{etoolbox}
\makeatletter
\patchcmd\longtable{\par}{\if@noskipsec\mbox{}\fi\par}{}{}
\makeatother
% Allow footnotes in longtable head/foot
\IfFileExists{footnotehyper.sty}{\usepackage{footnotehyper}}{\usepackage{footnote}}
\makesavenoteenv{longtable}
\usepackage{graphicx}
\makeatletter
\def\maxwidth{\ifdim\Gin@nat@width>\linewidth\linewidth\else\Gin@nat@width\fi}
\def\maxheight{\ifdim\Gin@nat@height>\textheight\textheight\else\Gin@nat@height\fi}
\makeatother
% Scale images if necessary, so that they will not overflow the page
% margins by default, and it is still possible to overwrite the defaults
% using explicit options in \includegraphics[width, height, ...]{}
\setkeys{Gin}{width=\maxwidth,height=\maxheight,keepaspectratio}
% Set default figure placement to htbp
\makeatletter
\def\fps@figure{htbp}
\makeatother

\makeatletter
\makeatother
\makeatletter
\makeatother
\makeatletter
\@ifpackageloaded{caption}{}{\usepackage{caption}}
\AtBeginDocument{%
\ifdefined\contentsname
  \renewcommand*\contentsname{Table of contents}
\else
  \newcommand\contentsname{Table of contents}
\fi
\ifdefined\listfigurename
  \renewcommand*\listfigurename{List of Figures}
\else
  \newcommand\listfigurename{List of Figures}
\fi
\ifdefined\listtablename
  \renewcommand*\listtablename{List of Tables}
\else
  \newcommand\listtablename{List of Tables}
\fi
\ifdefined\figurename
  \renewcommand*\figurename{Figure}
\else
  \newcommand\figurename{Figure}
\fi
\ifdefined\tablename
  \renewcommand*\tablename{Table}
\else
  \newcommand\tablename{Table}
\fi
}
\@ifpackageloaded{float}{}{\usepackage{float}}
\floatstyle{ruled}
\@ifundefined{c@chapter}{\newfloat{codelisting}{h}{lop}}{\newfloat{codelisting}{h}{lop}[chapter]}
\floatname{codelisting}{Listing}
\newcommand*\listoflistings{\listof{codelisting}{List of Listings}}
\makeatother
\makeatletter
\@ifpackageloaded{caption}{}{\usepackage{caption}}
\@ifpackageloaded{subcaption}{}{\usepackage{subcaption}}
\makeatother
\makeatletter
\@ifpackageloaded{tcolorbox}{}{\usepackage[skins,breakable]{tcolorbox}}
\makeatother
\makeatletter
\@ifundefined{shadecolor}{\definecolor{shadecolor}{rgb}{.97, .97, .97}}
\makeatother
\makeatletter
\makeatother
\makeatletter
\makeatother
\journal{Journal Name}
\ifLuaTeX
  \usepackage{selnolig}  % disable illegal ligatures
\fi
\usepackage[]{natbib}
\bibliographystyle{elsarticle-harv}
\IfFileExists{bookmark.sty}{\usepackage{bookmark}}{\usepackage{hyperref}}
\IfFileExists{xurl.sty}{\usepackage{xurl}}{} % add URL line breaks if available
\urlstyle{same} % disable monospaced font for URLs
\hypersetup{
  pdftitle={Quantile Regressions via Method of Moments with multiple fixed effects},
  pdfauthor={Fernando Rios-Avila; Leonardo Siles; Gustavo Canavire-Bacarreza},
  pdfkeywords={Fixed effects, Linear heteroskedasticity, Location-scale
model, Quantile regression},
  colorlinks=true,
  linkcolor={blue},
  filecolor={Maroon},
  citecolor={Blue},
  urlcolor={Blue},
  pdfcreator={LaTeX via pandoc}}

\setlength{\parindent}{6pt}
\begin{document}

\begin{frontmatter}
\title{Quantile Regressions via Method of Moments with multiple fixed
effects}
\author[1]{Fernando Rios-Avila%
\corref{cor1}%
}
 \ead{friosavi@levy.org} 
\author[]{Leonardo Siles%
%
}
 \ead{leo@gmail.com} 
\author[2]{Gustavo Canavire-Bacarreza%
%
}
 \ead{gcanavire@worldbank.org} 

\affiliation[1]{organization={Levy Economics Institute of Bard
College},,postcodesep={}}
\affiliation[2]{organization={World Bank},,postcodesep={}}

\cortext[cor1]{Corresponding author}



        
\begin{abstract}
This paper proposes a new method to estimate quantile regressions with
multiple fixed effects. The method expands on the strategy proposed by
\citet{mss2019}, allowing for multiple fixed effects, and providing
various alternatives for the estimation of Standard errors. We provide
Monte Carlo simulations to show the finite sample properties of the
proposed method in the presence of two sets of fixed effects. Finally,
we apply the proposed method to estimate \textbf{something interesting}
\end{abstract}





\begin{keyword}
    Fixed effects \sep Linear heteroskedasticity \sep Location-scale
model \sep 
    Quantile regression
\end{keyword}
\end{frontmatter}
    \ifdefined\Shaded\renewenvironment{Shaded}{\begin{tcolorbox}[interior hidden, frame hidden, enhanced, sharp corners, breakable, borderline west={3pt}{0pt}{shadecolor}, boxrule=0pt]}{\end{tcolorbox}}\fi

\hypertarget{introduction}{%
\section{Introduction}\label{introduction}}

Quantile regression (QR), introduced by Koenker and Bassett (1978), is
an estimation strategy used for modeling the relationships between
explanatory variables X and the conditional quantiles of the dependent
variable \(q_\tau (y|x)\). Using QR one can obtain richer
characterizations of the relationships between dependent and independent
variables, by accounting for otherwise unobserved heterogeneity.

A relatively recent development in the literature has focused on
extending quantile regressions analysis to include individual fixed
effects in the framework of panel data. However, as described in Neyman
and Scott (1948), and Lancaster (2000), when individual fixed effects
are included in quantile regression analysis it generates an incident
parameter problem. While many strategies have been proposed for
estimating this type of model (see Galvão and Kato, 2018 for a brief
review), neither has become standard because of their restrictive
assumptions in regards to the individual effects, the computational
complexity, and implementation.

More recently, \citet{mss2019} (MSS hereafter) proposed a methodology
based on a conditional location-scale model similar to the one described
in \citet{he1997}, for the estimation of quantile regressions models for
panel data via a method of moments. This method allows individual fixed
effects allowing to have heterogeneous effects on the entire conditional
distribution of the outcome, rather constraining their effect to be a
location shift only as in Canay (2011), Koenker(2004), and
Lancaster(2000).

In principle, under the assumption that data generating process behind
the data is based on a multiplicative heteroskedastic process that is
linear in parameters (Cameron and Trivedi, 2005; \citet{mss2019};
\citet{he1997}), the effect of a variable \(X\) on the \(q_th\) quantile
can be derived as the combination of a location effect, and scale effect
moderated by the quantile of an underlying i.i.d. error. For statistical
inference, MSS derives the asymptotic distribution of the estimator,
suggesting the use of bootstrap standard errors, as well.

While this methodology is not meant to substitute the use of standard
quantile regression analysis, given the assumptions required for the
identification of the model, it provides a simple and fast alternative
for the estimation of quantile regression models with individual fixed
effects.

In this framework, our paper expands on \citet{mss2019}, following some
of the suggestions by the authors regarding further research. First,
making use of the properties of GMM estimators, we derive various
alternatives for the estimation of standard errors based on the
empirical Influence functions of the estimators. Second, we reconsider
the application of Frisch--Waugh--Lovell (FWL) theorem (Lovell (1963)
and Frisch and Waugh's (1933)) to extend the MSS estimator to allow for
the inclusion of multiple fixed effects, for example, individual and
year fixed effects.

The rest of the paper is restructured as follows. Section 2 presents the
basic setup of the location-Scale model described in \citet{he1997},
tying the relationship between the standard quantile regression model,
and the location and scale model. It also revisits MSS methodology,
proposing alternative estimators for the standard errors based on the
properties of GMM estimators and the empirical influence functions. It
also shows that FWL theorem can be used to control for multiple fixed
effects. Section 3 presents the results of a small simulation study and
Section 4 illustrates the application of the proposed methods with two
empirical examples. Seccion 5 concludes.

\hypertarget{methodology}{%
\section{Methodology}\label{methodology}}

\hypertarget{quantile-regression-location-scale-model-he1997}{%
\subsection{\texorpdfstring{Quantile Regression Location-Scale model
\citet{he1997}}{Quantile Regression Location-Scale model @he1997}}\label{quantile-regression-location-scale-model-he1997}}

\hypertarget{standard-errors-gls-robust-clustered}{%
\subsection{Standard Errors: GLS, Robust,
Clustered}\label{standard-errors-gls-robust-clustered}}

\hypertarget{multiple-fixed-effects-expanding-on-mss2019}{%
\subsection{\texorpdfstring{Multiple Fixed Effects: Expanding on
\citet{mss2019}}{Multiple Fixed Effects: Expanding on @mss2019}}\label{multiple-fixed-effects-expanding-on-mss2019}}

\hypertarget{monte-carlo-simulations}{%
\section{Monte Carlo Simulations}\label{monte-carlo-simulations}}

\hypertarget{application-something-interesting}{%
\section{\texorpdfstring{Application: \textbf{Something
interesting}}{Application: Something interesting}}\label{application-something-interesting}}

\hypertarget{conclusions}{%
\section{Conclusions}\label{conclusions}}

\hypertarget{references}{%
\section*{References}\label{references}}
\addcontentsline{toc}{section}{References}


\renewcommand\refname{Appendix}
  \bibliography{bibliography.bib}


\end{document}
