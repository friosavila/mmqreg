% Options for packages loaded elsewhere
\PassOptionsToPackage{unicode}{hyperref}
\PassOptionsToPackage{hyphens}{url}
\PassOptionsToPackage{dvipsnames,svgnames,x11names}{xcolor}
%
\documentclass[
  letterpaper,
  DIV=11,
  numbers=noendperiod]{scrartcl}

\usepackage{amsmath,amssymb}
\usepackage{iftex}
\ifPDFTeX
  \usepackage[T1]{fontenc}
  \usepackage[utf8]{inputenc}
  \usepackage{textcomp} % provide euro and other symbols
\else % if luatex or xetex
  \usepackage{unicode-math}
  \defaultfontfeatures{Scale=MatchLowercase}
  \defaultfontfeatures[\rmfamily]{Ligatures=TeX,Scale=1}
\fi
\usepackage{lmodern}
\ifPDFTeX\else  
    % xetex/luatex font selection
\fi
% Use upquote if available, for straight quotes in verbatim environments
\IfFileExists{upquote.sty}{\usepackage{upquote}}{}
\IfFileExists{microtype.sty}{% use microtype if available
  \usepackage[]{microtype}
  \UseMicrotypeSet[protrusion]{basicmath} % disable protrusion for tt fonts
}{}
\makeatletter
\@ifundefined{KOMAClassName}{% if non-KOMA class
  \IfFileExists{parskip.sty}{%
    \usepackage{parskip}
  }{% else
    \setlength{\parindent}{0pt}
    \setlength{\parskip}{6pt plus 2pt minus 1pt}}
}{% if KOMA class
  \KOMAoptions{parskip=half}}
\makeatother
\usepackage{xcolor}
\setlength{\emergencystretch}{3em} % prevent overfull lines
\setcounter{secnumdepth}{-\maxdimen} % remove section numbering
% Make \paragraph and \subparagraph free-standing
\ifx\paragraph\undefined\else
  \let\oldparagraph\paragraph
  \renewcommand{\paragraph}[1]{\oldparagraph{#1}\mbox{}}
\fi
\ifx\subparagraph\undefined\else
  \let\oldsubparagraph\subparagraph
  \renewcommand{\subparagraph}[1]{\oldsubparagraph{#1}\mbox{}}
\fi


\providecommand{\tightlist}{%
  \setlength{\itemsep}{0pt}\setlength{\parskip}{0pt}}\usepackage{longtable,booktabs,array}
\usepackage{calc} % for calculating minipage widths
% Correct order of tables after \paragraph or \subparagraph
\usepackage{etoolbox}
\makeatletter
\patchcmd\longtable{\par}{\if@noskipsec\mbox{}\fi\par}{}{}
\makeatother
% Allow footnotes in longtable head/foot
\IfFileExists{footnotehyper.sty}{\usepackage{footnotehyper}}{\usepackage{footnote}}
\makesavenoteenv{longtable}
\usepackage{graphicx}
\makeatletter
\def\maxwidth{\ifdim\Gin@nat@width>\linewidth\linewidth\else\Gin@nat@width\fi}
\def\maxheight{\ifdim\Gin@nat@height>\textheight\textheight\else\Gin@nat@height\fi}
\makeatother
% Scale images if necessary, so that they will not overflow the page
% margins by default, and it is still possible to overwrite the defaults
% using explicit options in \includegraphics[width, height, ...]{}
\setkeys{Gin}{width=\maxwidth,height=\maxheight,keepaspectratio}
% Set default figure placement to htbp
\makeatletter
\def\fps@figure{htbp}
\makeatother
\newlength{\cslhangindent}
\setlength{\cslhangindent}{1.5em}
\newlength{\csllabelwidth}
\setlength{\csllabelwidth}{3em}
\newlength{\cslentryspacingunit} % times entry-spacing
\setlength{\cslentryspacingunit}{\parskip}
\newenvironment{CSLReferences}[2] % #1 hanging-ident, #2 entry spacing
 {% don't indent paragraphs
  \setlength{\parindent}{0pt}
  % turn on hanging indent if param 1 is 1
  \ifodd #1
  \let\oldpar\par
  \def\par{\hangindent=\cslhangindent\oldpar}
  \fi
  % set entry spacing
  \setlength{\parskip}{#2\cslentryspacingunit}
 }%
 {}
\usepackage{calc}
\newcommand{\CSLBlock}[1]{#1\hfill\break}
\newcommand{\CSLLeftMargin}[1]{\parbox[t]{\csllabelwidth}{#1}}
\newcommand{\CSLRightInline}[1]{\parbox[t]{\linewidth - \csllabelwidth}{#1}\break}
\newcommand{\CSLIndent}[1]{\hspace{\cslhangindent}#1}

\KOMAoption{captions}{tableheading}
\makeatletter
\makeatother
\makeatletter
\makeatother
\makeatletter
\@ifpackageloaded{caption}{}{\usepackage{caption}}
\AtBeginDocument{%
\ifdefined\contentsname
  \renewcommand*\contentsname{Table of contents}
\else
  \newcommand\contentsname{Table of contents}
\fi
\ifdefined\listfigurename
  \renewcommand*\listfigurename{List of Figures}
\else
  \newcommand\listfigurename{List of Figures}
\fi
\ifdefined\listtablename
  \renewcommand*\listtablename{List of Tables}
\else
  \newcommand\listtablename{List of Tables}
\fi
\ifdefined\figurename
  \renewcommand*\figurename{Figure}
\else
  \newcommand\figurename{Figure}
\fi
\ifdefined\tablename
  \renewcommand*\tablename{Table}
\else
  \newcommand\tablename{Table}
\fi
}
\@ifpackageloaded{float}{}{\usepackage{float}}
\floatstyle{ruled}
\@ifundefined{c@chapter}{\newfloat{codelisting}{h}{lop}}{\newfloat{codelisting}{h}{lop}[chapter]}
\floatname{codelisting}{Listing}
\newcommand*\listoflistings{\listof{codelisting}{List of Listings}}
\makeatother
\makeatletter
\@ifpackageloaded{caption}{}{\usepackage{caption}}
\@ifpackageloaded{subcaption}{}{\usepackage{subcaption}}
\makeatother
\makeatletter
\@ifpackageloaded{tcolorbox}{}{\usepackage[skins,breakable]{tcolorbox}}
\makeatother
\makeatletter
\@ifundefined{shadecolor}{\definecolor{shadecolor}{rgb}{.97, .97, .97}}
\makeatother
\makeatletter
\makeatother
\makeatletter
\makeatother
\ifLuaTeX
  \usepackage{selnolig}  % disable illegal ligatures
\fi
\IfFileExists{bookmark.sty}{\usepackage{bookmark}}{\usepackage{hyperref}}
\IfFileExists{xurl.sty}{\usepackage{xurl}}{} % add URL line breaks if available
\urlstyle{same} % disable monospaced font for URLs
\hypersetup{
  pdftitle={Quantile Regressions via Method of Moments with multiple fixed effects},
  pdfauthor={Fernando Rios-Avila; Leonardo Siles; Gustavo Canavire-Bacarreza},
  pdfkeywords={Fixed effects, Linear heteroskedasticity, Location-scale
model, Quantile regression},
  colorlinks=true,
  linkcolor={blue},
  filecolor={Maroon},
  citecolor={Blue},
  urlcolor={Blue},
  pdfcreator={LaTeX via pandoc}}

\title{Quantile Regressions via Method of Moments with multiple fixed
effects}
\author{Fernando Rios-Avila \and Leonardo Siles \and Gustavo
Canavire-Bacarreza}
\date{2023-10-23}

\begin{document}
\maketitle
\begin{abstract}
This paper proposes a new method to estimate quantile regressions with
multiple fixed effects. The method expands on the strategy proposed by
Machado and Santos Silva (2019), allowing for multiple fixed effects,
and providing various alternatives for the estimation of Standard
errors. We provide Monte Carlo simulations to show the finite sample
properties of the proposed method in the presence of two sets of fixed
effects. Finally, we apply the proposed method to estimate
\textbf{something interesting}
\end{abstract}
\ifdefined\Shaded\renewenvironment{Shaded}{\begin{tcolorbox}[interior hidden, enhanced, sharp corners, borderline west={3pt}{0pt}{shadecolor}, frame hidden, breakable, boxrule=0pt]}{\end{tcolorbox}}\fi

\hypertarget{introduction}{%
\section{Introduction}\label{introduction}}

Quantile regression (QR), introduced by Koenker and Bassett (1978), is
an estimation strategy used for modeling the relationships between
explanatory variables X and the conditional quantiles of the dependent
variable \(q_\tau (y|x)\). Using QR one can obtain richer
characterizations of the relationships between dependent and independent
variables, by accounting for otherwise unobserved heterogeneity.

A relatively recent development in the literature has focused on
extending quantile regressions analysis to include individual fixed
effects in the framework of panel data. However, as described in Neyman
and Scott (1948), and Lancaster (2000), when individual fixed effects
are included in quantile regression analysis it generates an incident
parameter problem. While many strategies have been proposed for
estimating this type of model (see Galvão and Kato, 2018 for a brief
review), neither has become standard because of their restrictive
assumptions in regards to the individual effects, the computational
complexity, and implementation.

More recently, Machado and Santos Silva (2019) (MSS hereafter) proposed
a methodology based on a conditional location-scale model similar to the
one described in He (1997) and (\textbf{zhao2020?}), for the estimation
of quantile regressions models for panel data via a method of moments.
This method allows individual fixed effects allowing to have
heterogeneous effects on the entire conditional distribution of the
outcome, rather constraining their effect to be a location shift only as
in Canay (2011), Koenker (2004), and Lancaster (2000).

In principle, under the assumption that data generating process behind
the data is based on a multiplicative heteroskedastic process that is
linear in parameters (Cameron and Trivedi (2005); Machado and Santos
Silva (2019); He (1997); (\textbf{zhao2020?})), the effect of a variable
\(X\) on the \(q_th\) quantile can be derived as the combination of a
location effect, and scale effect moderated by the quantile of an
underlying i.i.d. error. For statistical inference, MSS derives the
asymptotic distribution of the estimator, suggesting the use of
bootstrap standard errors, as well.

While this methodology is not meant to substitute the use of standard
quantile regression analysis, given the assumptions required for the
identification of the model, it provides a simple and fast alternative
for the estimation of quantile regression models with individual fixed
effects.

In this framework, our paper expands on Machado and Santos Silva (2019),
following some of the suggestions by the authors regarding further
research. First, making use of the properties of GMM estimators, we
derive various alternatives for the estimation of standard errors based
on the empirical Influence functions of the estimators. Second, we
reconsider the application of Frisch--Waugh--Lovell (FWL) theorem
(Frisch and Waugh (1933) and Lovell (1963)) to extend the MSS estimator
to allow for the inclusion of multiple fixed effects, for example,
individual and year fixed effects.

The rest of the paper is restructured as follows. Section 2 presents the
basic setup of the location-Scale model described in He (1997) and
(\textbf{zhao2020?}), tying the relationship between the standard
quantile regression model, and the location and scale model. It also
revisits MSS methodology, proposing alternative estimators for the
standard errors based on the properties of GMM estimators and the
empirical influence functions. It also shows that FWL theorem can be
used to control for multiple fixed effects. Section 3 presents the
results of a small simulation study and Section 4 illustrates the
application of the proposed methods with two empirical examples. Seccion
5 concludes.

\hypertarget{methodology}{%
\section{Methodology}\label{methodology}}

\hypertarget{quantile-regression-location-scale-model}{%
\subsection{Quantile Regression: Location-Scale
model}\label{quantile-regression-location-scale-model}}

Quantile regressions are used to identify relationships between the
explanatory variables \(x\) and the conditional quantiles of the
dependent variable \(Q(y|\tau,X)\). This relationship is commonly
assumed to follow a linear functional form:

\begin{equation}\protect\hypertarget{eq-eq1}{}{Q(Y|X,\tau) =X\beta(\tau)
}\label{eq-eq1}\end{equation}

This allows for nonlinearities in the effect of \(X\) on \(Y\) across
all values of \(\tau\). This formulation can also be related to a random
coefficient model, where all coefficients are assumed to be some
nonlinear function of \(\tau\), where \(\tau\) follows a random uniform
distribution.

An alternative formulation of quantile regressions is the location-scale
model. This approach assumes that the conditional quantile of \(Y\)
given \(X\) and \(\tau\) can be expressed as a combination of two
models: the location model, which describes the central tendency of the
conditional distribution, and the scale model, which describes
deviations from the central tendency:

\begin{equation}\protect\hypertarget{eq-eq2}{}{Q(Y|X,\tau) =X\beta+X\gamma(\tau)
}\label{eq-eq2}\end{equation}

Here, the location parameters \(\beta\) are typically identified using a
linear regression model (as in Machado and Santos Silva (2019)), or a
median regression (as in Melly (2005)), and the scale parameters
\(\gamma(\tau)\) can be estimated using standard approaches.

Both the standard quantile regression (Equation~\ref{eq-eq1}) and the
location-scale specification (Equation~\ref{eq-eq2}) can be estimated as
the solution to a weighted minimization problem:

\begin{equation}\protect\hypertarget{eq-eq3}{}{\hat{\beta}(\tau) = \underset{\beta}{\operatorname{argmin}}
\left( \sum_{i\in y_i\geq x_i'\beta} \tau (y_i - x_i'\beta) - \sum_{i\in y_i<x_i'\beta} (1-\tau)(y_i - x_i'\beta) \right)
}\label{eq-eq3}\end{equation}

One characteristic of this estimator is that the \(\beta(\tau)\)
coefficients are identified locally, and thus the estimated quantile
coefficients will exhibit considerable variation when analyzed across
\(\tau\). It is also implicit that if one requires an analysis of the
entire distribution, it would be necessary to estimate the model for
each quantile.\footnote{There are other estimators that provide smoother
  estimates for the quantile regression coefficients using a kernel
  local weighted approach (Kaplan and Sun 2017), as well as identifying
  the full set of quantile coefficients simultaneously assuming some
  parametric functional forms (Frumento and Bottai 2016).}

One insightful extension to the location-scale parameterizations
suggested by He (1997), Cameron and Trivedi (2005), and Machado and
Santos Silva (2019) is to assume that the data-generating process (DGP)
can be written as a linear model with a multiplicative heteroskedastic
process that is linear in parameters.\footnote{Machado and Santos Silva
  (2019) also discuss a model where heteroskedasticity can be an
  arbitrary nonlinear function \(\sigma(x_i'\gamma)\), but develop the
  estimator for the linear case, i.e., when \(\sigma()\) is the identity
  function.}

\begin{equation}\protect\hypertarget{eq-eq4}{}{y_i=x_i'\beta+\varepsilon_i \times x_i'\gamma(\tau)
}\label{eq-eq4}\end{equation}

Under the assumption that \(\varepsilon\) is an independent and
identically distributed (iid) unobserved random variable that is
independent of \(X\), the conditional quantile of \(Y\) given \(X\) and
\(\tau\) can be written as:

\begin{equation}\protect\hypertarget{eq-eq5}{}{Q(Y|X,\tau) =X\beta+X\gamma \times Q(\varepsilon|\tau)
}\label{eq-eq5}\end{equation}

In this setup, the traditional quantile coefficients are identified as
the location model coefficients, plus the scale model coefficients
moderated by the \(\tau_th\) unconditional quantile of the standardized
error \(\varepsilon\).

\begin{equation}\protect\hypertarget{eq-eq6}{}{\beta(\tau) = \beta + \gamma \times Q(\varepsilon|\tau)
}\label{eq-eq6}\end{equation}

While this specification imposes a strong assumption on the DGP, it has
two advantages over the standard quantile regression model. First,
because the location and scale model can be identified globally, with
only a single parmater (\(Q(\varepsilon|\tau)\)) requiring local
estimation, this estimation approach would be more efficient than the
standard quantile regression model (Zhao (2000)). Second, under the
assumption that \(X\gamma\) is strictly possitive, the model would
produce quantile coefficients that do not cross.

Following MSS, the quantile regression model defined by
Equation~\ref{eq-eq5} can be estimated using a method of moments
approach. And while its possible to identify all coefficients
(\(\beta,\gamma, Q(\varepsilon|\tau)\)) simultaneously, we describe and
use the implementation approach advocated by MSS which identifies each
set of coefficients separately.

\begin{enumerate}
\def\labelenumi{\arabic{enumi}.}
\item
  The location model can be estimated using a standard linear regression
  model, where the dependent variable is the outcome \(Y\), and the
  independent variables are the explanatory variables \(X\) (including a
  constant) with an error \(u\), which is by definition heteroskedastic.
  In this case, the location model coefficients are identified under the
  following condition:
  \begin{equation}\protect\hypertarget{eq-eq7}{}{\begin{aligned}
    y_i &=x_i'\beta+u_i \\
    E[u 'X ] &=0
    \end{aligned}
  }\label{eq-eq7}\end{equation}
\item
  After the location model is estimated, the scale coefficients can be
  identified by modeling heteroskedasticity as a linear function of
  characteristics \(X\). For this we use the absolute value of the
  errors from the location model \(u\) as dependent variable, which
  would allow us to estimate the conditional standard deviation (rather
  than conditional variance) of the errors. In this case, the
  coefficients are identified under the following condition:
\end{enumerate}

\begin{equation}\protect\hypertarget{eq-eq8}{}{\begin{aligned}
  |u_i| &=x_i'\gamma+v_i \\
  E[ (|u|-X \gamma) X ] &=0
  \end{aligned}
}\label{eq-eq8}\end{equation}

\begin{enumerate}
\def\labelenumi{\arabic{enumi}.}
\setcounter{enumi}{2}
\tightlist
\item
  Finally, given the location and scale coefficients, the \(\tau_{th}\)
  quantile of the error \(e\) can be estimated using the following
  condition:
\end{enumerate}

\begin{equation}\protect\hypertarget{eq-eq9}{}{\begin{aligned}
  E\left[  \mathbb{1}\left(x_i' (\beta + \gamma Q(\varepsilon|\tau)) \geq y_i \right) - \tau \right] &=0  \\
  E\left[  \mathbb{1}\left(   Q(\varepsilon|\tau)\geq \frac{y_i-x_i'\beta}{x_i'\gamma} \right) - \tau \right] &=0  \\
  \end{aligned}
}\label{eq-eq9}\end{equation}

Where one identifies the quantile of the error \(\varepsilon\) using
standardized errors \(\frac{y_i-x_i'\beta}{x_i'\gamma}\), or by finding
the values that identify the overall quantile coefficients
\(\beta(\tau)=\beta + \gamma Q(\varepsilon|\tau)\). Afterwords, the
conditional quantile coefficients is simply defined as the combination
of the location and scale coefficients.

\hypertarget{standard-errors-gls-robust-clustered}{%
\subsection{Standard Errors: GLS, Robust,
Clustered}\label{standard-errors-gls-robust-clustered}}

As discussed in the previous section, the estimation of quantile
regression coefficients using the location-scale model with
heteroskedstic linear errors can be estimated using a the following set
of moments, which fits in the Generalized Method of Moments framework:

\begin{equation}\protect\hypertarget{eq-eq10}{}{\begin{aligned}
  E[u_i x_i ] &= E[h_{1,i}]=0 \\
  E[ (|u_i|-x_i \gamma) x_i ] &=E[h_{2,i}]=0 \\
  E\left[  \mathbb{1}\left(   Q(\varepsilon|\tau)\geq \frac{y_i-x_i'\beta}{x_i'\gamma} \right) - \tau \right] 
  &=E[h_{3,i}]=0 
  \end{aligned}
}\label{eq-eq10}\end{equation}

Under the conditions described in Newey and McFadden (1994) (see section
7), Cameron and Trivedi (2005) (see chapter 6.3.9) or as shown in
Machado and Santos Silva (2019), the location, scale and residual
quantile coefficients are asymptotically normal.\footnote{Zhao (2000)
  also shows that the quantile coefficients for the location-scale model
  also follows a normal distribution, but uses the assumption that the
  location model is derived using a least absolute deviation approach
  (median regression).}

Call \(\theta=[ \beta' \ \ \gamma' \ \ Q(\varepsilon|\tau)' ]'\) the set
of coefficients that are identified in Equation~\ref{eq-eq10}, a just
identified model. And the functions \(G_{k,l}\) and \(S_{k.l}\) equal
to:

\[\begin{aligned}
G_{k,l} &= N^{-1} \sum_{i=1}^N \frac{\partial h_{k,i}}{\partial \theta_l} \Big|_{\hat\theta} \ \forall k,l \in [1,2,3] \\
S_{k,l} &= N^{-1} \sum_{i=1}^N h_{k,i} h_{l,i}' \Big|_{\hat\theta} \ \forall k,l \in [1,2,3]
\end{aligned}
\]

where \(\theta_1 = \beta\), \(\theta_2 = \gamma\) and
\(\theta_3 = Q(\varepsilon,\tau)\).

Then \[
\sqrt{N}(\hat\theta - \theta) \xrightarrow{d} N(0, G^{-1} S G^{-1})
\]

with, \(G\) and \(S\) defined as:

\[\begin{aligned}
G &= \begin{bmatrix}
G_{1,1} &       0 & 0       \\
G_{2,1} & G_{2,2} & 0       \\
G_{3,1} & G_{3,2} & G_{3,3} 
\end{bmatrix}  \text{ and }
S &= \begin{bmatrix}
S_{1,1} & S_{1,2} & S_{1,3}    \\
S_{2,1} & S_{2,2} & S_{2,3}  \\
S_{3,1} & S_{3,2} & S_{3,3} 
\end{bmatrix}
\end{aligned}
\]

Because

\[
Var(\hat\theta) = \frac{1}{N} \left(
    \frac{1}{N} 
    \sum_{i=1}^N  \lambda_i(\theta) \lambda_i(\theta)'  
\right) = \frac{1}{N^2} 
\lambda(\theta)'\lambda(\theta)
\]

\[
\lambda_i(\theta) = \bar G(\theta)^{-1} h_i (x_i,\theta)
\]

Finally, the variance covariance of \(\hat \beta(\tau)\) will be given
by:

\[Var(\hat\beta(\tau)) = \Xi Var(\hat\theta) \Xi'\]

where \(\Xi\) is a \(k \times (2k+1)\) matrix defined as:

\[
\begin{aligned}
    \Xi &= I(k), \hat q(\tau) I(k), \hat \gamma \\
    \Xi &= \left[ \begin{array}{cccc|cccc|}
           1 & 0 & \dots & 0 &\hat q_(\tau) & 0 & ... & 0  & \hat\gamma_0 \\
           0 & 1 & \dots & 0 &0& \hat q_(\tau) & ... & 0 & \hat\gamma_1 \\
          \vdots & \vdots & \dots & 0 &...&...& ... & 0             & ...\\
           0 & 0 & \dots & 1 &0 & 0 & ... & \hat q(\tau)   & \hat\gamma_k
        \end{array} \right]         
\end{aligned}
\]

\hypertarget{multiple-fixed-effects-expanding-on-mss2019}{%
\subsection{Multiple Fixed Effects: Expanding on Machado and Santos
Silva (2019)}\label{multiple-fixed-effects-expanding-on-mss2019}}

\hypertarget{monte-carlo-simulations}{%
\section{Monte Carlo Simulations}\label{monte-carlo-simulations}}

\hypertarget{application-something-interesting}{%
\section{\texorpdfstring{Application: \textbf{Something
interesting}}{Application: Something interesting}}\label{application-something-interesting}}

\hypertarget{conclusions}{%
\section{Conclusions}\label{conclusions}}

\hypertarget{appendix}{%
\section{Appendix}\label{appendix}}

\hypertarget{references}{%
\section*{References}\label{references}}
\addcontentsline{toc}{section}{References}

\hypertarget{refs}{}
\begin{CSLReferences}{1}{0}
\leavevmode\vadjust pre{\hypertarget{ref-cameron2005}{}}%
Cameron, Adrian Colin, and Pravin K. Trivedi. 2005.
\emph{Microeconometrics: Methods and Applications}. Cambridge ; New
York: Cambridge University Press.

\leavevmode\vadjust pre{\hypertarget{ref-canay2011}{}}%
Canay, Ivan A. 2011. {``A Simple Approach to Quantile Regression for
Panel Data.''} \emph{The Econometrics Journal} 14 (3): 368--86.
https://doi.org/\url{https://doi.org/10.1111/j.1368-423X.2011.00349.x}.

\leavevmode\vadjust pre{\hypertarget{ref-frishwaugh1933}{}}%
Frisch, Ragnar, and Frederick V. Waugh. 1933. {``Partial Time
Regressions as Compared with Individual Trends.''} \emph{Econometrica} 1
(4): 387--401. \url{http://www.jstor.org/stable/1907330}.

\leavevmode\vadjust pre{\hypertarget{ref-frumentobotai2016}{}}%
Frumento, Paolo, and Matteo Bottai. 2016. {``Parametric Modeling of
Quantile Regression Coefficient Functions.''} \emph{Biometrics} 72 (1):
74--84. https://doi.org/\url{https://doi.org/10.1111/biom.12410}.

\leavevmode\vadjust pre{\hypertarget{ref-he1997}{}}%
He, Xuming. 1997. {``Quantile Curves Without Crossing.''} \emph{The
American Statistician} 51 (2): 186--92.
\url{https://doi.org/10.1080/00031305.1997.10473959}.

\leavevmode\vadjust pre{\hypertarget{ref-kaplan2017}{}}%
Kaplan, David M., and Yixiao Sun. 2017. {``Smoothed Estimating Equations
for Instrumental Variables Quantile Regression.''} \emph{Econometric
Theory} 33 (1): 105--57.
\url{https://doi.org/10.1017/S0266466615000407}.

\leavevmode\vadjust pre{\hypertarget{ref-koenker2004}{}}%
Koenker, Roger. 2004. {``Quantile Regression for Longitudinal Data.''}
\emph{Journal of Multivariate Analysis} 91 (1): 74--89.
https://doi.org/\url{https://doi.org/10.1016/j.jmva.2004.05.006}.

\leavevmode\vadjust pre{\hypertarget{ref-koenkerbasset1978}{}}%
Koenker, Roger, and Gilbert Bassett. 1978. {``Regression Quantiles.''}
\emph{Econometrica} 46 (1): 33--50.
\url{http://www.jstor.org/stable/1913643}.

\leavevmode\vadjust pre{\hypertarget{ref-lancaster2000}{}}%
Lancaster, Tony. 2000. {``The Incidental Parameter Problem Since
1948.''} \emph{Journal of Econometrics} 95 (2): 391--413.
https://doi.org/\url{https://doi.org/10.1016/S0304-4076(99)00044-5}.

\leavevmode\vadjust pre{\hypertarget{ref-lovell1963}{}}%
Lovell, Michael C. 1963. {``Seasonal Adjustment of Economic Time Series
and Multiple Regression Analysis.''} \emph{Journal of the American
Statistical Association} 58 (304): 993--1010.
\url{https://doi.org/10.1080/01621459.1963.10480682}.

\leavevmode\vadjust pre{\hypertarget{ref-mss2019}{}}%
Machado, José A. F., and J. M. C. Santos Silva. 2019. {``Quantiles via
Moments.''} \emph{Journal of Econometrics} 213 (1): 145--73.
\url{https://doi.org/10.1016/j.jeconom.2019.04.009}.

\leavevmode\vadjust pre{\hypertarget{ref-melly2005}{}}%
Melly, Blaise. 2005. {``Decomposition of Differences in Distribution
Using Quantile Regression.''} \emph{Labour Economics} 12 (4): 577--90.
https://doi.org/\url{https://doi.org/10.1016/j.labeco.2005.05.006}.

\leavevmode\vadjust pre{\hypertarget{ref-newey_chapter_1994}{}}%
Newey, Whitney K., and Daniel McFadden. 1994. {``Chapter 36 {Large}
Sample Estimation and Hypothesis Testing.''} In \emph{Handbook of
{Econometrics}}, 4:2111--2245. Elsevier.
\url{https://doi.org/10.1016/S1573-4412(05)80005-4}.

\leavevmode\vadjust pre{\hypertarget{ref-neymanscott1948}{}}%
Neyman, J., and Elizabeth L. Scott. 1948. {``Consistent Estimates Based
on Partially Consistent Observations.''} \emph{Econometrica} 16 (1):
1--32. \url{http://www.jstor.org/stable/1914288}.

\leavevmode\vadjust pre{\hypertarget{ref-zhao2000}{}}%
Zhao, Quanshui. 2000. {``Restricted {Regression} {Quantiles}.''}
\emph{Journal of Multivariate Analysis} 72 (1): 78--99.
https://doi.org/\url{https://doi.org/10.1006/jmva.1999.1849}.

\end{CSLReferences}



\end{document}
